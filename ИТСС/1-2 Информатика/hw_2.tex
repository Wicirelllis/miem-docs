\documentclass[a4paper,12pt]{article}
\usepackage{ucs}
\usepackage{cmap}
\usepackage[utf8x]{inputenc}
\usepackage{amsfonts}
\usepackage[english,russian]{babel}
\usepackage[T1,T2A]{fontenc}
\frenchspacing
\usepackage{amsmath,amssymb,amsthm}
\usepackage[a4paper, margin=1in]{geometry}
\usepackage[table]{xcolor}
\usepackage{multirow}
\usepackage{diagbox}
\usepackage{graphicx}
\graphicspath{ {./images/} }
\usepackage{pgfplots}
\usepgfplotslibrary{fillbetween}

\begin{document}
\renewcommand{\arraystretch}{1.5}




\large
\begin{titlepage}
\begin{center}
ФЕДЕРАЛЬНОЕ ГОСУДАРСТВЕННОЕ АВТОНОМНОЕ ОБРАЗОВАТЕЛЬНОЕ УЧРЕЖДЕНИЕ ВЫСШЕГО ОБРАЗОВАНИЯ «НАЦИОНАЛЬНЫЙ ИССЛЕДОВАТЕЛЬСКИЙ УНИВЕРСИТЕТ «ВЫСШАЯ ШКОЛА ЭКОНОМИКИ»

\vspace{1cm}

Московский институт электроники и математики им. А.Н. Тихонова

\vspace{2cm}

Ефремов Виктор Васильевич, группа БИТ 203

\vspace{4cm}

Отчет

по домашней работе 1, часть 3
\\

по дисциплине "Информатика"

Тема: "Математические основы вычислительной техники"

\vspace{1cm}

Номер варианта: 6

Дата сдачи отчета: 30.10.2020

\vfill

Москва
2020
\end{center}
\end{titlepage}
\normalsize




\setcounter{page}{2}

\section*{Предисловие}

Напомним числа из варианта №6.

$N = 29$

$M = 4$

$p_1 = 0.87$

$p_2 = 0.36$

$p_3 = 0.94$
\\

Будем решать задачи в общем виде и подставлять числа в самом конце.
Потому что задачи у всех одинаковые, только числа различаются ;).
\\

Предполагается, что читатель слышал про сочетания, размещения и перестановки, и их версии с повторениями.
А также правила сложения и умножения.

\newpage




\section*{3.1}

\textit{Студенческая группа состоит из 33 человек, среди которых $N$ юношей и $M$ девушек (в соответствии с вариантом).
Сколькими способами можно выбрать двух человек одного пола?}
\\

Можно выбрать либо двух девушек, либо двух парней.

Посмотрим сколькими способами можно выбрать пару юношей.

Порядок людей в паре не важен, поэтому это сочетание из $N$ по $2$, равное $$C_N^2 = \frac{N!}{(N - 2)! \cdot 2!} = \frac{N \cdot (N - 1)}{2}$$

Но можно и руками посчитать.
Первым можно взять любого, поэтому $N$ вариантов.
Вторым - любого из оставшихся, т.е. $N - 1$ вариантов.
Т.е. всего $N \cdot (N - 1)$ пар.
При этом мы посчитали каждую пару дважды, взяв первым человеком пары сначала одного, а потом другого.
И т.к. порядок людей в паре не важен, нужно поделить произведение пополам.
Итого способов вырать пару юношей: $\frac{N \cdot (N - 1)}{2}$

Аналогично для девушек: \[\frac{M \cdot (M - 1)}{2}\]

Складываем, подставляем числа, получаем итоговый ответ: \[\frac{N \cdot (N - 1) + M \cdot (M - 1)}{2} = \frac{29 \cdot 28 + 4 \cdot 3}{2} = 412\]




\section*{3.2}

\textit{Согласно государственному стандарту пароль должен состоять из N букв и M цифр.
Буквы при этом могут быть как верхнего, так и нижнего регистров.
При этом недопустимы пароли, в которых все буквы одинаковые (независимо от регистра).
Сколько различных паролей может быть использовать согласно стандарту?}
\\

Будем считать, что буквы берутся из английского алфавита и их 26 (от a до z), а цифр 10 (от 0 до 9).

Заметим, что каждый пароль однозначно определятся местами букв в пароле, упорядоченным набором $N$ букв и упорядоченным набором $M$ цифр.

Поясним на примере.
Пусть пароль 1tt01e.
Буквы находятся на 2, 3, 6 местах.
Упорядоченный набор букв получается из пароля выбрасыванием всех цифр - tte.
Набор цифр - 101.

Посчитаем места букв.
Нужно выбрать $N$ мест среди $N + M$ мест.
Их, очевидно, $C_{N + M}^{N} = \frac{(N + M)!}{N! \cdot M!}$.

Посчитаем наборы букв.
Т.к. мы используем и строчные, и прописные буквы, то просто слово длины $N$ можно составить $52^{N}$ способами.
%Из этого количества нужно выкинуть слова из одинаковых букв, которых $26 \cdot 2^N$

Посчитаем слова из одинаковых букв.
Например из \textbf{A} и \textbf{a}.
Первую букву можно выбрать из двух вариантов (\textbf{A} и \textbf{a}), вторую тоже, третью тоже, и т.д.
Поэтому всего таких слов $2^N$.
А т.к. вместо \textbf{A} и \textbf{a} можно брать любые из 26 букв, то всего нужно выкинуть $26 \cdot 2^N$.

Наборы цифр аналогичны наборам букв.
Их ${10}^{M}$.

Итоговый ответ: \[\frac{(N + M)!}{N! \cdot M!} \cdot (52^{N} - 26 \cdot 2^N) \cdot {10}^{M}\]

Подставляем числа: \[\frac{33!}{29! \cdot 4!} \cdot (52^{29} - 26 \cdot 2^{29}) \cdot 10^4 \approx 2.4 \cdot 10^{58}\]




\section*{3.3}

\textit{С трех различных рабочих станций на сервер передаются пакеты данных.
Вероятность успешной передачи пакета данных от первой рабочей станции $p_1$, от второй - $p_2$, от третьей - $p_3$.
Найти вероятность того, что:}

\textit{а. хотя бы один пакет данных будет принят сервером;}

\textit{б. только два пакета будут приняты сервером;}

\textit{в. сервер примет не менее двух пакетов.}
\\

а. Рассмотрим отрицание утверждения.
"Неверно, что хотя бы один пакет данных будет принят сервером" $\Leftrightarrow$ "Не будет принято ни одного пакета".
Его вероятность $(1 - p_1) \cdot (1 - p_2) \cdot (1 - p_3)$.
Вероятности перемножаются, т.к. события независимы - потеря или прием первого пакета не влияют на второй и третий.

Отсюда искомая вероятность:
\[1 - (1 - p_1) \cdot (1 - p_2) \cdot (1 - p_3)\]
Подставляем числа:
\[1 - (1 - 0.87) \cdot (1 - 0.36) \cdot (1 - 0.94) = 0.995008\]

б. Просто рассмотрим все варианты.
Вероятность того, что первый и второй пакеты приняты, а третий потерян - $p_1 \cdot p_2 \cdot (1 - p_3)$.
Аналогично принятые первый и третий - $p_1 \cdot (1 - p_2) \cdot p_3$.
Принятые второй и третий - $(1 - p_1) \cdot p_2 \cdot p_3$.

Итого ответ:
\[p_1 \cdot p_2 \cdot (1 - p_3) + p_1 \cdot (1 - p_2) \cdot p_3 + (1 - p_1) \cdot p_2 \cdot p_3\]
Подставляем числа:
\[0.87 \cdot 0.36 \cdot (1 - 0.94) + 0.87 \cdot (1 - 0.36) \cdot 0.94 + (1 - 0.87) \cdot 0.36 \cdot 0.94 = 0.586176\]

в. "Сервер примет не менее двух пакетов" $\Leftrightarrow$ "Сервер примет либо два, либо три пакета".
Вероятность двух пакетов посчитана в пункте б.
Вероятность принять все три пакета - $p_1 \cdot p_2 \cdot p_3$.

Поэтому ответ:
\[p_1 \cdot p_2 \cdot (1 - p_3) + p_1 \cdot (1 - p_2) \cdot p_3 + (1 - p_1) \cdot p_2 \cdot p_3 + p_1 \cdot p_2 \cdot p_3\]
Подставляем числа:
\[0.87 \cdot 0.36 \cdot (1 - 0.94) + 0.87 \cdot (1 - 0.36) \cdot 0.94 + (1 - 0.87) \cdot 0.36 \cdot 0.94 + 0.87 \cdot 0.36 \cdot 0.94 = 0.880584\]




\section*{3.4}

\textit{Студент сдаёт экзамен по теории вероятностей, при этом $N$ билетов он знает хорошо, а $M$ плохо.
Предположим, в первый день экзамен сдаёт часть группы, например, $\frac{N}{2}$ человек, включая нашего героя.
В каком случае студенту с большей вероятностью достанется «хороший» билет – если он пойдёт «в первых рядах», или если зайдёт «посерединке», или если будет тянуть билет в числе последних?
Когда лучше заходить?}

Неважно.
Вероятность вытащить хороший билет всегда одинакова.

Пусть наш герой идет первым.
Тогда вероятность взять хороший билет равна
\[\frac{N}{N + M}\]

Рассмотрим теперь случай, когда герой идет вторым.
Первый человек может вытащить либо плохой билет с вероятностью $\frac{M}{N + M}$, либо хороший с вероятностью $\frac{N}{N + M}$.
В первом случае шансы героя повышаются, вероятность получить хороший билет становится $\frac{N}{N + M - 1}$.
Во втором уменьшаются - $\frac{N - 1}{N + M - 1}$.

Итого, шансы вытащить хороший билет, заходя вторым, равны

\begin{gather*}
\frac{M}{N + M} \cdot \frac{N}{N + M - 1} + \frac{N}{N + M} \cdot \frac{N - 1}{N + M - 1} = \frac{N \cdot M + N^2 - N}{(N + M)(N + M -1)} =\\
= \frac{N \cdot (N + M - 1)}{(N + M)(N + M - 1)} = \frac{N}{N + M}
\end{gather*}

Таким образом мы показали, что разницы между первым и вторым местами нет.
Но в рассуждениях выше важно лишь что эти два человека идут друг за другом.
А первый и второй они, или семнадцатый и восемнадцатый не важно.
Т.е. соседние места равнозначны.

А следовательно любые два места равны, т.к. их всегда можно соеденить цепочкой соседних.




\section*{3.5}

\textit{Генератор чисел выдает на выходе два числа $x$ и $y$ в промежутке от $0$ до $(p_1 + p_2 + p_3) \cdot 100$.
Какова вероятность, что $y > \frac{x^2}{2} - 1$?}
\\

Прежде всего положим, что генератор выдает равномерно распределенные и независимые действительные числа.

Также положим $p = (p_1 + p_2 + p_3) \cdot 100$.
\\

В этой задаче крйне полезна наглядная/геометрическая интерпретация.
Схеатически построим график $f(x) = \frac{x^2}{2} - 1$ при $x \in [0; p], y \in [0, p]$.
Для простоты и нагядности будем счиитать $p = 10$.

\begin{center}
\begin{tikzpicture}
\begin{axis}[
  axis equal image,
  axis lines = middle,
  xlabel = $x$,
  ylabel = $y$,
  xmin = 0,
  xmax = 10,
  domain = 0:10, 
  ymin = 0,
  ymax = 10
  ]
\addplot [thick, samples=100, color=black, name path=A] {x^2/2 - 1};
\addplot [samples=50, smooth, black, name path=B] coordinates {(0,10)(10,10)};
\addplot [samples=50, smooth, black, name path=C] coordinates {(0,0)(10,0)};
\addplot [samples=50, smooth, black] coordinates {(10,0)(10,10)};
\addplot [samples=50, smooth, dashed, black] coordinates {(4.69,0)(4.69,10)};
\addplot [green!30] fill between [of=A and B, soft clip={domain=0:10}];
\addplot [red!30] fill between [of=A and C, soft clip={domain=0:10}];
\end{axis}
\end{tikzpicture}
\end{center}

Неравенство из условия выполняется для пары $(x, y)$ только если эта точка лежит выше графика, т.е. попадает в зеленую область.
Соответственно, не выполняется, если точка лежит ниже графика.
И поэтому искомая вероятность будет равна
\[\frac{S_{a}}{S_{a} + S_{b}} = 1 - \frac{S_b}{S}\]
где $S_a$ - площадь над графиком (зеленая область), $S_b$ - площадь под графиком (красная), $S$ - площадь квадрата на котором мы строим график.
\\

Ясно, что $S = p^2$
\\

Площадь $S_b$ можно посчитать по-разному, например взяв интеграл.

$f(x_1) = 0 \Rightarrow x_1 = \sqrt{2}$

$f(x_2) = p \Rightarrow x_2 = \sqrt{2p + 2}$

\begin{gather*}
S_b = \int\displaylimits_{x_1}^{x_2} f(x) \, dx + (p - x_2) \cdot p = \int\displaylimits_{\sqrt{2}}^{\sqrt{2p + 2}} \left( \frac{x^2}{2} - 1 \right) dx + \left( p - \sqrt{2p + 2} \right) \cdot p = \\
= \left. \left( \frac{x^3}{6} - x \right) \right|_{\sqrt{2}}^{\sqrt{2p+2}} + \left( p - \sqrt{2p + 2} \right) \cdot p = \\
%= \sqrt{2p+2} * (\frac{2p+2}{6} - 1) - \sqrt{2}/3 + \sqrt{2} + (p - \sqrt{2p + 2}) * p = \\
= \left( \frac{\left( \sqrt{2p+2} \right)^3}{6} - \sqrt{2p+2} \right) - \left( \frac{ \left( \sqrt{2} \right)^3}{6} - \sqrt{2} \right) + \left( p - \sqrt{2p + 2} \right) \cdot p = \\
= p^2 - \frac{2}{3} \cdot \sqrt{2} \cdot \left( \left( \sqrt{p + 1} \right)^3 - 1 \right)
\end{gather*}

Поэтому искомая вероятность:
% - это отношение количества (или в нашем случае площади) благоприятных/хороших исходов ко всем.

%Итоговый ответ:
\begin{gather*}
\frac{2}{3} \cdot \sqrt{2} \cdot \frac{ \left( \sqrt{p + 1} \right)^3 - 1}{p^2}
\end{gather*}

Подставим числа:
\begin{gather*}
p = (0.87 + 0.36 + 0.94) \cdot 100 = 217 \\
\frac{2}{3} \cdot \sqrt{2} \cdot \frac{ \left( \sqrt{217 + 1} \right)^3 - 1}{217^2} \approx 0.064
\end{gather*}




\end{document}
