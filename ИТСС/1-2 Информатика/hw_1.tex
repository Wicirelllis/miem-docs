\documentclass[a4paper,12pt]{article}
\usepackage{ucs}
\usepackage[utf8x]{inputenc}
\usepackage{amsfonts}
\usepackage[english,russian]{babel}
\usepackage[T1,T2A]{fontenc}
\frenchspacing
\usepackage{amsmath,amssymb,amsthm}
\usepackage[a4paper, margin=1in]{geometry}
\usepackage[table]{xcolor}
\usepackage{multirow}
\usepackage{diagbox}
\usepackage{graphicx}
\graphicspath{ {./images/} }

\begin{document}
\renewcommand{\arraystretch}{1.5}




\large
\begin{titlepage}
\begin{center}
ФЕДЕРАЛЬНОЕ ГОСУДАРСТВЕННОЕ АВТОНОМНОЕ ОБРАЗОВАТЕЛЬНОЕ УЧРЕЖДЕНИЕ ВЫСШЕГО ОБРАЗОВАНИЯ «НАЦИОНАЛЬНЫЙ ИССЛЕДОВАТЕЛЬСКИЙ УНИВЕРСИТЕТ «ВЫСШАЯ ШКОЛА ЭКОНОМИКИ»

\vspace{1cm}

Московский институт электроники и математики им. А.Н. Тихонова

\vspace{2cm}

Ефремов Виктор Васильевич, группа БИТ 203

\vspace{4cm}

Отчет
по домашней работе 1

по дисциплине "Информатика"
Тема: "Математические основы вычислительной техники"

\vspace{1cm}

Номер варианта: 6

Дата сдачи отчета: 18.10.2020

\vfill

Москва
2020
\end{center}
\end{titlepage}
\normalsize



\setcounter{page}{2}
\section*{1.1}

\textit{Перевести числа в двоичное представление. Из него в восьмиричное и шестнадцатиричное. Сделать проверку.}
\\

$\begin{array}{c @{\, = \,} r @{.} l}
a & 496 & 82 \\
b & 100 & 635 \\
c & 606 & 274 \\
d & 747 & 33 \\
e & 380 & 12 \\
f & 552 & 806 \\
\end{array}$
\\

Сделаем только первое число с комментариями и пояснениями, т.к. алгоритм одинаков для всех.
Будем переводить отдельно целую часть, отдельно дробную.

Рассмотрим первое число $a = 496.82$.

Переведем его целую часть в двоичную систему счисления.
Для этого будем делить число нацело на 2 и записывать остатки.
Искомое двоичное представление - это остатки записанные в обратном порядке.

$\begin{array}{r @{\, = \,}r@{\, \cdot \, 2 \, + \,}l}
496 & 248 & 0 \\
248 & 124 & 0 \\
124 & 62 & 0 \\
62 & 31 & 0 \\
31 & 15 & 1 \\
15 & 7 & 1 \\
7 & 3 & 1 \\
3 & 1 & 1 \\
1 & 0 & 1 \\
\end{array}$

\[496_{10} = 111110000_2\]

Такой способ записи, хотя и ясный, слишком громоздкий.
Поэтому в дальнейшем будем просто записывать числа в две колонки: слева частное, справа остаток.

$\begin{array}{r l}
496 \\
248 & 0 \\
124 & 0 \\
62 & 0 \\
31 & 0 \\
15 & 1 \\
7 & 1 \\
3 & 1 \\
1 & 1 \\
0 & 1 \\
\end{array}$
\\

Для перевода дробной части умножаем её на 2 и записываем целую часть результата (0 или 1).
Повторяем до тех пор, пока дробная часть не станет нулем, или до достижения нужной точности.
Целые части взятые в прямом порядке - двоичное представление дробной части.

Замечание про точность и знаки после точки.
Чем мньше основание системы счисления, тем больше знаков нужно для той же точности.
Если мы переводим из q-ичные числа в p-ичные, то на каждый знак после точки в исходном числе нужно $log_p q$ знаков в результате.
Например для перевода из десятичной в двоичную на каждый знак десятичный нужно $log_2 10 \approx 3.3219$ двоичных знаков.
Таким образом для двух знаков после точки в десятичной записи нужно 7 (округляем в большую сторону) в двоичной, а для трех - 10.

$\begin{array}{r @{\, \cdot \, 2 \, = \,} l @{\, + \,} l}
0.82 & 0.64 & 1 \\
0.64 & 0.28 & 1 \\
0.28 & 0.56 & 0 \\
0.56 & 0.12 & 1 \\
0.12 & 0.24 & 0 \\
0.24 & 0.48 & 0 \\
0.48 & 0.96 & 0 \\
\end{array}$

\[0.82_{10} \approx 1101000_2\]

Эта запись также громоздка, поэтому будем записывать только последовательные результаты умножения дробной части на $2$:

$0.82 - 1.64 - 1.28 - 0.56 - 1.12 - 0.24 - 0.48 - 0.96$
\\

Итого 
\[496.82_{10} \approx 111110000.1101000_2\]

Стоит помнить, что это неточное представление. Оно лишь имеет сходную точность (примерно до сотой).
\\

Переведем теперь число из двоичной в восьмиричную и шестнадцатиричную систеы счисления.
Для этого разобьем цифры по 3 (т.к. $8 = 2^3$) и 4 ($16 = 2^4$) начиная от разделителя-точки и дописывая незначащие нули, если цифр не хватает.

$111\,110\,000.110\,100\,000_2 = 760.64_8$

$0001\,1111\,0000.1101\,0000_2 = 1F0.D0_{16}$


Проверку сделаем просто любым калькулятором/конвертером.

Например https://binary2hex.ru/numberconverter.html

\newpage




$b = 100.635$
\\

$\begin{array}{r l}
100 & \\
50 & 0 \\
25 & 0 \\
12 & 1 \\
6 & 0 \\
3 & 0 \\
1 & 1 \\
0 & 1 \\
\end{array}$

$100_{10} = 1100100_2$
\\

$0.635 - 1.270 - 0.540 - 1.080 - 0.160 - 0.320 - 0.640 - 1.280 - 0.560 - 1.120 - 0.240$

$0.635_{10} \approx 1010001010_2$
\\

$100.635_{10} \approx 1100100.1010001010_2$

$001 \, 100 \, 100.101 \, 000 \, 101 \, 000_2 = 144.5050_8$

$0110 \, 0100.1010 \, 0010 \, 1000_2 = 64.A28_{16}$

\newpage





$c = 606.274$
\\

$\begin{array}{r l}
606 & \\
303 & 0 \\
151 & 1 \\
75 & 1 \\
37 & 1 \\
18 & 1 \\
9 & 0 \\
4 & 1 \\
2 & 0 \\
1 & 0 \\
0 & 1 \\
\end{array}$

$606_{10} = 1001011110_2$
\\

$0.274 - 0.548 - 1.096 - 0.192 - 0.384 - 0.768 - 1.536 - 1.072 - 0.144 - 0.288 - 0.576$

$0.274_{10} \approx 0.0100011000_2$
\\

$606.274_{10} \approx 1001011110.0100011000_2$

$001 \, 001 \, 011 \, 110.010 \, 001 \, 100 \, 000_2 = 1136.2140_8$

$0010 \, 0101 \, 1110.0100 \, 0110 \, 0000_2 = 25E.460_{16}$

\newpage




$d = 747.33$
\\

$\begin{array}{r l}
747 & \\
373 & 1 \\
186 & 1 \\
93 & 0 \\
46 & 1 \\
23 & 0 \\
11 & 1 \\
5 & 1 \\
2 & 1 \\
1 & 0 \\
0 & 1 \\
\end{array}$

$747_{10} = 1011101011_2$
\\

$0.33 - 0.66 - 1.32 - 0.64 - 1.28 - 0.56 - 1.12 - 0.24$

$0.33_{10} \approx 0.0101010_2$
\\

$747.33_{10} \approx 1011101011.0101010_2$

$001 \, 011 \, 101 \, 011.010 \, 101 \, 000_2 = 1353.25_8$

$0010 \, 1110 \, 1011.0101 \, 0100_2 = 2EB.54_{16}$

\newpage




$e = 380.12$
\\

$\begin{array}{r l}
380 & \\
190 & 0 \\
95 & 0 \\
47 & 1 \\
23 & 1 \\
11 & 1 \\
5 & 1 \\
2 & 1 \\
1 & 0 \\
0 & 1 \\
\end{array}$

$380_{10} = 101111100_2$
\\

$0.12 - 0.24 - 0.48 - 0.96 - 1.92 - 1.84 - 1.68 - 1.36$

$0.12_{10} \approx 0001111_2$
\\

$380.12_{10} \approx 101111100.0001111_2$

$101 \, 111 \, 100.000 \, 111 \, 100_2 = 574.074_8$

$0001 \, 0111 \, 1100.0001 \, 1110_2 = 17C.1E_{16}$

\newpage




$f = 552.806$
\\

$\begin{array}{r l}
552 & \\
276 & 0 \\
138 & 0 \\
69 & 0 \\
34 & 1 \\
17 & 0 \\
8 & 1 \\
4 & 0 \\
2 & 0 \\
1 & 0 \\
0 & 1 \\
\end{array}$

$552_{10} = 1000101000_2$
\\

$0.806 - 1.612 - 1.224 - 0.448 - 0.896 - 1.792 - 1.584 - 1.168 - 0.336 - 0.672 - 1.344$

$0.806_{10} \approx 0.1100111001_2$
\\

$552.806_{10} \approx 1000101000.1100111001_2$

$001 \, 000 \, 101 \, 000.110 \, 011 \, 100 \, 100_2 = 1050.6344_8$

$0010 \, 0010 \, 1000.1100 \, 1110 \, 0100_2 = 228.CE8_{16}$

\newpage




\section*{1.2}

\textit{Выполнить умножение и деление двоичных чисел. Сделать проверку.}
\\

$X_1 = a \cdot b = 496.82 \cdot 100.635 \approx 111110000.1101000 \cdot 1100100.1010001010$

Для простоты умножения отбросим незначащие нули, а также точку-разделитель.

\begin{verbatim}
                    1111100001101 000
                 1100100101000101 0
--------------------------------------
                    1111100001101
                  1111100001101
              1111100001101
            1111100001101
         1111100001101
      1111100001101
     1111100001101
--------------------------------------
    11000011010011001001110000001 0000
 
    1100001101001100.10011100000010000
\end{verbatim}

$X_1 \approx 1100001101001100.10011100000010000$

Проверка:

$X_1 - a \cdot b \approx 49996.6094970703125_{10} - 496.82_{10} \cdot 100.635_{10} \approx -0.87120292968$

Стоит отметить что погрешность $X_1$ на уровне единиц.
\\

$X_3 = c \cdot d = 606.274 \cdot 747.33 \approx 1001011110.0100011000 \cdot 1011101011.0101010$

\begin{verbatim}
                 10010111100100011 000
                  1011101011010101 0
--------------------------------------
                 10010111100100011
               10010111100100011
             10010111100100011
           10010111100100011
          10010111100100011
        10010111100100011
      10010111100100011
     10010111100100011
    10010111100100011
  10010111100100011
--------------------------------------
  11011101001110111010011000011111 0000

  1101110100111011101.00110000111110000
\end{verbatim}

$X_3 \approx 1101110100111011101.00110000111110000$

Проверка:

$X_3 - c \cdot d \approx 453085.1912841796875_{10} - 606.274_{10} \cdot 747.33_{10} \approx -1.55713582033$

\newpage




\section*{1.3}

\textit{Перевести данное число в десятичное из а) беззнакового б) дополнительного кода}
\\

$A88D_{16} = 1010 \, 1000 \, 1000 \, 1101_2$
\\

а) Число без знака, поэтому просто складываем степени двойки:
\[2^{15} + 2^{13} + 2^{11} + 2^7 + 2^3 + 2^2 + 2^0 = 32768 + 8192 + 2048 + 128 + 8 + 4 + 1 = 43149\]

Ответ: $43149$.
\\

б) Число со знаком, дополнительный код.
Самый старший (левый) бит - это знак, все остальное это модуль.
При этом нужно инвертировать биты и прибавить единицу, т.к. у нас отрицательное число.
Инверсия $010 \, 1000 \, 1000 \, 1101_2$ - это $101 \, 0111 \, 0111 \, 0010_2$, прибавляем единицу $101 \, 0111 \, 0111 \, 0011_2$, переводим в десятичную систему счисления:

\[2^{14} + 2^{12} + 2^{10} + 2^9 + 2^8 + 2^6 + 2^5 + 2^4 + 2^1 + 2^0 = 22387\]

Ответ: $-22387$.

\newpage




\section*{1.4}

\textit{Записать числа в прямом, обратном и дополнительном кодах. Выполнить сложение. Отметить переполнения.}
\\

Запишем все числа для удобства в таблицу.

Во всех трех кодах первый/старший бит - это знак.
Оставшиеся биты представляют модуль числа.

Стоит помнить, что все три кода совпадают на положительных числах.

Для отрицательных чисел и нуля алгоритм следующий.

Чтобы получить прямой код просто переводим число из десятичного в двоичное.
Добавляем незначащих ведущих нулей при необходимости.
Не забываем про единицу знака в старшем бите.

Модуль числа в обратном коде получается из прямого инверсией бит.

Дополнительный код получается из обратного добавлением единицы.
\\

\begin{center}
$\begin{array}{| c | c | c | c | c |}
\hline
& \text{десятичное} & \text{прямой} & \text{обратный} & \text{дополнительный} \\ \hline
a & 496 & 00000001 \, 11110000 & 00000001 \, 11110000 & 00000001 \, 11110000 \\ \hline
b & 100 & 00000000 \, 01100100 & 00000000 \, 01100100 & 00000000 \, 01100100 \\ \hline
c & 606 & 00000010 \, 01011110 & 00000010 \, 01011110 & 00000010 \, 01011110 \\ \hline
d & 747 & 00000010 \, 11101011 & 00000010 \, 11101011 & 00000010 \, 11101011 \\ \hline
e & -380 & 10000001 \, 01111100 & 11111110 \, 10000011 & 11111110 \, 10000100 \\ \hline
f & -552 & 10000010 \, 00101000 & 11111101 \, 11010111 & 11111101 \, 11011000 \\ \hline
\end{array}$
\end{center}

Выполним сложение.

$X_5 = a + b$

Сложим их в прямом коде.
Т.к. оба положительны, никаких дополнительных действий делать не нужно.

\begin{verbatim}
    0000 0001 1111 0000
+   0000 0000 0110 0100
-----------------------
    0000 0010 0101 0100
\end{verbatim}

$X_5 = 0000 0010 \, 0101 0100$

\newpage

$X_6 = c + e$

Эти два сложим в обратном коде.
Т.к. у нас в сумме появляется единица, которая не влезает в 16 бит, мы переносим её направо и складываем с ответом.

\begin{verbatim}
    0000 0010 0101 1110
+   1111 1110 1000 0011
-----------------------
  1 0000 0000 1110 0001
+                     1
-----------------------
    0000 0000 1110 0010
\end{verbatim}

$X_6 = 0000 0000 \, 1110 0010$
\\

$X_7 = d + f$

А это сложим в дополнительном коде.
В дополнительном коде вылезающая за границы единица просто отбрасывается.

\begin{verbatim}
    0000 0010 1110 1011
+   1111 1101 1101 1000
-----------------------
  1 0000 0000 1100 0011
-----------------------
    0000 0000 1100 0011
\end{verbatim}

$X_7 = 0000 0000 \, 1100 0011$

\newpage




\section*{1.5}

\textit{Записать числа в float из стандарта IEEE 754. Произвести сложение. Сделать проверку.}
\\

Биты в флоате распределены следующим образом: 1 для знака, 8 для порядка/экспоненты, 23 для мантиссы.\
\\

Для представления числа в IEEE 754 формате удобно предварительно записать его в виде

\[(-1)^{s} \cdot 2^{e} \cdot m\]

Где $s$ - знак, $e$ - экспонента, $m$ - мантисса.
\\

Стоит помнить, что экспонента записывается в коде со сдвигом 127.
Мантисса не меньше единицы и меньше двух, т.е. $1 \leq m < 2$.
Из-за этого целая часть мантиссы всегда равна 1 и не записывается.
\\

Ещё один неочевидный момент - это округление мантиссы.
По-умолчанию используется "round to nearest, ties to even" (до ближайшего; до четного, если ближайшие равноудалены).

Рассмотрим такое округление на нескольких примерах.
Все числа в примерах двоичные.

Пусть мы хотим округлить $100.1$ до целого.
Кандидатов быть ответом два: $100$ и $101$.
Но $100.1$ в \emph{точности} по середене между $100$ и $101$, поэтому мы округляем к четному, т.е. числу с нулем на конце.
Ответ $100$.

Округлим число $101.1000000000010$ до целого.
Кандидатов опять же два: $101$ и $110$.
$101.1000000000010$ ближе к $110$.
Действительно, $110 - 101.1000000000010 = 0.0111111111110$, $101 - 101.1000000000010 = -0.1000000000010$.
Первое расстояние меньше $0.1_2 = 0.5_{10}$, а второе больше.
Поэтому ответ - $110$.
\\

Запишем число $a = 496.82 \approx 111110000.1101000$.

Найденое в 1.1 представление недостаточно точное.
Нам нужно больше знаков после точки.
Посчитаем.

$0.82 - 1.64 - 1.28 - 0.56 - 1.12 - 0.24 - 0.48 - 0.96 - 1.92 - 1.84 - 1.68 - 1.36 - 0.72 - 1.44 - 0.88 - 1.76 - 1.52 - 1.04 - 0.08 - 0.16 - 0.32 - 0.64$

$a \approx 111110000.110100011110101110000_2 = (-1)^0 \cdot 1.11110000110100011110101 \, 110000_2 \cdot 10_2^{1000_2}$

Знак $s = 0$, т.к. число положительное.

Экспонента равна $8_{10} = 1000_2$, но т.к. мы её записываем в коде со сдвигом, к ней нужно прибавить $127_{10} = 111 \, 1111_2$.
Поэтому $e = 1000 0111$.

Посмотрим на мантиссу.
Нам нужно 23 двоичных знака после точки.
Но просто отрезать хвост нельзя, т.к. по стандарту мантисса округляется "to nearest, ties to even".
Поэтому мантисса будет равна $m = 111 1000 0110 1000 1111 0101$.
\\

Таким образом число $a$ записывается как

\[a_{float} = 01000011111110000110100011110110\]

%\newpage




$b = 100.635$

$0.635 - 1.270 - 0.540 - 1.080 - 0.160 - 0.320 - 0.640 - 1.280 - 0.560 - 1.120 - 0.240 - 0.480 - 0.960 - 1.920 - 1.840 - 1.680 - 1.360 - 0.720 - 1.440 - 0.880 - 1.760 - 1.520 - 1.040 - 0.080$

$b \approx 1100100.10100010100011110101110_2 = 1.10010010100010100011110101110 \cdot 10^{110}$

$s = 0$

$e = 1000 0101$

$m = 10010010100010100011111$

\[b_{float} = 01000010110010010100010100011111\]




$c = 606.274$% = 0b1001011110.0100011000

$0.274 - 0.548 - 1.096 - 0.192 - 0.384 - 0.768 - 1.536 - 1.072 - 0.144 - 0.288 - 0.576 - 1.152 - 0.304 - 0.608 - 1.216 - 0.432 - 0.864 - 1.728 - 1.456 - 0.912 - 1.824 - 1.648 - 1.296 - 0.592$

$c \approx 1001011110.01000110001001001101110 = 1.00101111001000110001001001101110 \cdot 10^{1001}$

$s = 0$

$e = 10001000$

$m = 00101111001000110001001$

\[c_{float} = 01000100000101111001000110001001\]




$d = 747.33$

$0.33 - 0.66 - 1.32 - 0.64 - 1.28 - 0.56 - 1.12 - 0.24 - 0.48 - 0.96 - 1.92 - 1.84 - 1.68 - 1.36 - 0.72 - 1.44 - 0.88 - 1.76 - 1.52 - 1.04 - 0.08 - 0.16$

$d \approx 1011101011.010101000111101011100 = 1.011101011010101000111101011100 \cdot 10^{1001}$

$s = 0$

$e = 10001000$

$m = 01110101101010100011111$

\[d_{float} = 01000100001110101101010100011111\]




$e = 380.12$


$0.12 - 0.24 - 0.48 - 0.96 - 1.92 - 1.84 - 1.68 - 1.36 - 0.72 - 1.44 - 0.88 - 1.76 - 1.52 - 1.04 - 0.08 - 0.16 - 0.32 - 0.64 - 1.28 - 0.56 - 1.12 - 0.24n$

$e \approx 101111100.000111101011100001010 = 1.01111100000111101011100001010 \cdot 10^{1000}$

$s = 0$

$e = 10000111$

$m = 01111100000111101011100$

\[e_{float} = 01000011101111100000111101011100\]




$f = 552.806$

$0.806 - 1.612 - 1.224 - 0.448 - 0.896 - 1.792 - 1.584 - 1.168 - 0.336 - 0.672 - 1.344 - 0.688 - 1.376 - 0.752 - 1.504 - 1.008 - 0.016 - 0.032 - 0.064 - 0.128 - 0.256 - 0.512 - 1.024 - 0.048$

$f \approx 1000101000.11001110010101100000010 = 1.00010100011001110010101100000010 \cdot 10^{1001}$

$s = 0$

$e = 10001000$

$m = 00010100011001110010110$

\[f_{float} = 01000100000010100011001110010110\]

\newpage

Сложим числа.

$X_8 = a + c = 496.82 + 606.274$.

$a_{float} = 1.11110000110100011110110 \cdot 10^{1000}$

$c_{float} = 1.00101111001000110001001 \cdot 10^{1001}$

Для сложения чисел нужно приравнять их экспоненты, сложить мантиссы и нормализовать результат.

В нашем случае приведем оба числа к экспоненте $1001$ и сложим.

\begin{verbatim}
    0.111110000110100011110110
+   1.00101111001000110001001
------------------------------
   10.001001111000110000001000
\end{verbatim}

$10.001001111000110000001000 \cdot 10^{1001} = 1.0001001111000110000001000 \cdot 10^{1010}$

$X_8 = 0 10001001 00010011110001100000010 = 1103.093994140625$

Проверка:

$496.82 + 606.274 - 1103.093994140625 = 0.00000585937$
\\

$X_9 = b + e = 100.635 + 380.12$

$b_{float} = 1.10010010100010100011111 \cdot 10^{110}$

$e_{float} = 1.01111100000111101011100 \cdot 10^{1000}$

Приведем к экспоненте $1000$:

\begin{verbatim}
    0.0110010010100010100011111
+   1.01111100000111101011100
------------------------------
    1.1110000011000001010001111
\end{verbatim}

$X_9 = 0 10000111 11100000110000010100100 = 480.7550048828125$

Проверка:

$100.635 + 380.12 - 480.7550048828125 = -0.00000488281$
\\

$X_{10} = d + f = 747.33 + 552.806$

$d_{float} = 1.01110101101010100011111 \cdot 10^{1001}$

$f_{float} = 1.00010100011001110010110 \cdot 10^{1001}$

Здесь экспоненты уже совпадают.

\begin{verbatim}
    1.01110101101010100011111
+   1.00010100011001110010110
------------------------------
   10.10001010000100010110101
\end{verbatim}

$10.10001010000100010110101 \cdot 10^{1001} = 1.010001010000100010110101 \cdot 10^{1010}$

$X_{10} = 0 10001001 01000101000010001011010$

Проверка:

$747.33 + 552.806 - 1300.135986328125 = 0.00001367187$

\newpage



\section*{2.1}

$f_1 = \{0, 1, 4, 6, 7, 8, 9, 10, 11, 12, 13, 15\}$

Запишем таблицу истинности для $f$:

\begin{center}
	\begin{tabular}{| c | c |}
	\hline
	abcd & f \\
	\hline
	0000 & 1 \\
	0001 & 1 \\
	0010 & 0 \\
	0011 & 0 \\
	0100 & 1 \\
	0101 & 0 \\
	0110 & 1 \\
	0111 & 1 \\
	1000 & 1 \\
	1001 & 1 \\
	1010 & 1 \\
	1011 & 1 \\
	1100 & 1 \\
	1101 & 1 \\
	1110 & 0 \\
	1111 & 1 \\
	\hline
	\end{tabular}
\end{center}




\subsection*{2.1.1}

\textit{Записать СДНФ и СКНФ.}

Как мы помним, ДНФ - это сумма произведений (или дизъюнкция конъюнкций).
СДНФ (совершенная ДНФ) содержит только те термы, на которых функция равна 1.
Поэтому СДНФ это:

\[\overline{a}\overline{b}\overline{c}\overline{d} + \overline{a}\overline{b}\overline{c}d + \overline{a}b\overline{c}\overline{d} + \overline{a}bc\overline{d} + \overline{a}bcd + a\overline{b}\overline{c}\overline{d} + a\overline{b}\overline{c}d + a\overline{b}c\overline{d} + a\overline{b}cd + ab\overline{c}\overline{d} + ab\overline{c}d + abcd\]

 Аналогично, КНФ - это произведение сумм (или конъюнкция дизъюнкций).
В СКНФ используются термы на которых функция равна 0.
СКНФ:

\[(a + b + \overline{c} + d)(a + b + \overline{c} + \overline{d})(a + \overline{b} + c + \overline{d})(\overline{a} + \overline{b} + \overline{c} + d)\]




\subsection*{2.1.2}

\textit{Минимизировать функцию с помощью метода Квайна для СДНФ.}

Выпишем все наборы, на которых $f = 1$ и пронумеруем их.

Проведем операцию \emph{склейки}.
Для этого ищем строки таблицы различающиеся только в одной переменной и склеиваем их.
Записываем результат в соседнюю таблицу, а сами строки помечаем, например серым фоном.

Например термы $\overline{a}\overline{b}\overline{c}\overline{d}$ и $\overline{a}\overline{b}\overline{c}d$ различаются только в переменной $d$.
Поэтому их можно склеить.

Делаем это для всех возможных пар.

Повторяем процесс для второй и последующих таблиц.

\begin{table}[h!]
\centering
\begin{tabular}[t]{| c | c |}
\hline
\rowcolor{lightgray}1 & $\overline{a}\overline{b}\overline{c}\overline{d}$ \\
\rowcolor{lightgray}2 & $\overline{a}\overline{b}\overline{c}d$ \\
\rowcolor{lightgray}3 & $\overline{a}b\overline{c}\overline{d}$ \\
\rowcolor{lightgray}4 & $\overline{a}bc\overline{d}$ \\
\rowcolor{lightgray}5 & $\overline{a}bcd$ \\
\rowcolor{lightgray}6 & $a\overline{b}\overline{c}\overline{d}$ \\
\rowcolor{lightgray}7 & $a\overline{b}\overline{c}d$ \\
\rowcolor{lightgray}8 & $a\overline{b}c\overline{d}$ \\
\rowcolor{lightgray}9 & $a\overline{b}cd$ \\
\rowcolor{lightgray}10 & $ab\overline{c}\overline{d}$ \\
\rowcolor{lightgray}11 & $ab\overline{c}d$ \\
\rowcolor{lightgray}12 & $abcd$ \\
\hline
\end{tabular}
\quad
\begin{tabular}[t]{| c | c |}
\hline
\rowcolor{lightgray}1 - 2 & $\overline{a}\overline{b}\overline{c}$ \\
\rowcolor{lightgray}1 - 3 & $\overline{a}\overline{c}\overline{d}$ \\
\rowcolor{lightgray}1 - 6 & $\overline{b}\overline{c}\overline{d}$ \\
\rowcolor{lightgray}2 - 7 & $\overline{b}\overline{c}d$ \\
3 - 4 & $\overline{a}b\overline{d}$ \\
\rowcolor{lightgray}3 - 10 & $b\overline{c}\overline{d}$ \\
4 - 5 & $\overline{a}bc$ \\
5 - 12 & $bcd$ \\
\rowcolor{lightgray}6 - 7 & $a\overline{b}\overline{c}$ \\
\rowcolor{lightgray}6 - 8 & $a\overline{b}\overline{d}$ \\
\rowcolor{lightgray}6 - 10 & $a\overline{c}\overline{d}$ \\
\rowcolor{lightgray}7 - 9 & $a\overline{b}d$ \\
\rowcolor{lightgray}7 - 11 & $a\overline{c}d$ \\
\rowcolor{lightgray}8 - 9 & $a\overline{b}c$ \\
\rowcolor{lightgray}9 - 12 & $acd$ \\
\rowcolor{lightgray}10 - 11 & $ab\overline{c}$ \\
\rowcolor{lightgray}11 - 12 & $abd$ \\
\hline
\end{tabular}
\quad
\begin{tabular}[t]{| c | c |}
\hline
1 - 2 - 6 - 7 & $\overline{b}\overline{c}$ \\
1 - 3 - 6 - 10 & $\overline{c}\overline{d}$ \\
1 - 6 - 2 - 7 & $\overline{b}\overline{c}$ \\
1 - 6 - 3 - 10 & $\overline{c}\overline{d}$ \\
6 - 7 - 8 - 9 & $a\overline{b}$ \\
6 - 7 - 10 - 11 & $a\overline{c}$ \\
6 - 8 - 7 - 9 & $a\overline{b}$ \\
6 - 10 - 7 - 11 & $a\overline{c}$ \\
7 - 9 - 11 - 12 & $ad$ \\
7 - 11 - 9 - 12 & $ad$ \\
\hline
\end{tabular}
\end{table}

Посмотрим на не выделенные строки в таблицах выше.
В них записаны термы, которые ни с чем не склеились.
Это \emph{простые импликанты}.
Минимальная ДНФ - это сумма некотоорых (возможно всех) простых импликант.

Чтобы выяснить какие импликанты нужно брать в миинимальную форму заполним так называемую импликантную матрицу.
Протые импликанты записываются по строкам, слагаемые из СДНФ по столбцам.

Если импликанта поглощает терм СДНФ (т.е. целиком входит в него), то помечаем пересечение соответствующих строки и столбца (например галочкой).

Найдем галочки, которые единственные в своем столбце.
Выделим их.

Строки, в которых есть выделенные галочки - ядро.
Все импликанты составляющие ядро обязательно будут в минимальной форме.
В таблице ядро выделено зеленым.

Видно, что ядром остались не покрыты 6 столбцов.
Если взять в дополнение к ядру строки 3, 5 и 7, то получим минимальную ДНФ.

\begin{table}[h!]
\centering
	\begin{tabular}{| c | *{12}{c} |}
	\hline
	& \cellcolor{pink} $\overline{a}\overline{b}\overline{c}\overline{d}$ & \cellcolor{pink} $\overline{a}\overline{b}\overline{c}d$ & $\overline{a}b\overline{c}\overline{d}$ & $\overline{a}bc\overline{d}$ & $\overline{a}bcd$ & \cellcolor{pink} $a\overline{b}\overline{c}\overline{d}$ & \cellcolor{pink} $a\overline{b}\overline{c}d$ & \cellcolor{pink} $a\overline{b}c\overline{d}$ & \cellcolor{pink} $a\overline{b}cd$ & $ab\overline{c}\overline{d}$ & $ab\overline{c}d$ & $abcd$ \\
	\rowcolor{lime} $a\overline{b}$ & & & & & & $\checkmark$ & $\checkmark$ & \textcircled{$\checkmark$} & $\checkmark$ & & & \\
	$a\overline{c}$ & & & & & & $\checkmark$ & $\checkmark$ & & & $\checkmark$ & $\checkmark$ &\\
	$ad$ & & & & & & & $\checkmark$ & & $\checkmark$ & & $\checkmark$ & $\checkmark$ \\
	\rowcolor{lime} $\overline{b}\overline{c}$ & $\checkmark$ & \textcircled{$\checkmark$} & & & & $\checkmark$ & $\checkmark$ & & & & & \\
	$\overline{c}\overline{d}$ & $\checkmark$ & & $\checkmark$ & & & $\checkmark$ & & & & $\checkmark$ & & \\

	$\overline{a}b\overline{d}$ & & & $\checkmark$ & $\checkmark$ & & & & & & & & \\
	$\overline{a}bc$ & & & & $\checkmark$ & $\checkmark$ & & & & & & & \\
	$bcd$ & & & & & $\checkmark$ & & & & & & & $\checkmark$ \\

	\hline
	\end{tabular}
\end{table}

Таким образом в результате минимизации получается следующая ДНФ:

\[a\overline{b} + \overline{b}\overline{c} + ad + \overline{c}\overline{d} + \overline{a}bc\]

Чтобы убедиться, что это действительно МДНФ, нужно перебрать остальные варианты и проверить, что в них больше вхождений переменных.




\subsection*{2.1.3}

\textit{Минимизировать функцию с помощью диаграммы Вейча для СДНФ.}\\

Заполним диаграмму Вейча в соответствии с таблицей истинности.
Как мы помним, диаграмма заполняется зигзагами, начиная с правого нижнего угла.

\begin{center}
\begin{tabular}{c *{4}{|c}| c}
\multicolumn{1}{c}{} & \multicolumn{2}{c}{b} \\ \cline{2-5}
\multirow{2}{*}{a} & 1 & 1 & 1 & 1 \\ \cline{2-5}
& 0 & 1 & 1 & 1 & \multirow{2}{*}{c} \\ \cline{2-5}
& 1 & 1 & 0 & 0 \\ \cline{2-5}
& 1 & 0 & 1 & 1 \\ \cline{2-5}
\multicolumn{2}{c}{} & \multicolumn{2}{c}{d}
\end{tabular}
\end{center}


Посмотрим на элемент $\overline{a}\overline{b}\overline{c}d$.
Его можно накрыть четыремя вариантами, которые изображены ниже.

Заметим, что покрытие из четырех клеток включает в себя три других варианта.
Это значит, что эта группа попадает в ядро.

Ей соответствует терм $\overline{b}\overline{c}$.

\begin{tabular}{c *{4}{|c}| c}
\multicolumn{1}{c}{} & \multicolumn{2}{c}{b} \\ \cline{2-5}
\multirow{2}{*}{a} & 1 & 1 & \cellcolor{pink}1 & \cellcolor{pink}1 \\ \cline{2-5}
& 0 & 1 & 1 & 1 & \multirow{2}{*}{c} \\ \cline{2-5}
& 1 & 1 & 0 & 0 \\ \cline{2-5}
& 1 & 0 & \cellcolor{red}1 & \cellcolor{pink}1 \\ \cline{2-5}
\multicolumn{2}{c}{} & \multicolumn{2}{c}{d}
\end{tabular}
\quad
\begin{tabular}{c *{4}{|c}| c}
\multicolumn{1}{c}{} & \multicolumn{2}{c}{b} \\ \cline{2-5}
\multirow{2}{*}{a} & 1 & 1 & 1 & 1 \\ \cline{2-5}
& 0 & 1 & 1 & 1 & \multirow{2}{*}{c} \\ \cline{2-5}
& 1 & 1 & 0 & 0 \\ \cline{2-5}
& 1 & 0 & \cellcolor{red}1 & \cellcolor{pink}1 \\ \cline{2-5}
\multicolumn{2}{c}{} & \multicolumn{2}{c}{d}
\end{tabular}
\begin{tabular}{c *{4}{|c}| c}
\multicolumn{1}{c}{} & \multicolumn{2}{c}{b} \\ \cline{2-5}
\multirow{2}{*}{a} & 1 & 1 & \cellcolor{pink}1 & 1 \\ \cline{2-5}
& 0 & 1 & 1 & 1 & \multirow{2}{*}{c} \\ \cline{2-5}
& 1 & 1 & 0 & 0 \\ \cline{2-5}
& 1 & 0 & \cellcolor{red}1 & 1 \\ \cline{2-5}
\multicolumn{2}{c}{} & \multicolumn{2}{c}{d}
\end{tabular}
\quad
\begin{tabular}{c *{4}{|c}| c}
\multicolumn{1}{c}{} & \multicolumn{2}{c}{b} \\ \cline{2-5}
\multirow{2}{*}{a} & 1 & 1 & 1 & 1 \\ \cline{2-5}
& 0 & 1 & 1 & 1 & \multirow{2}{*}{c} \\ \cline{2-5}
& 1 & 1 & 0 & 0 \\ \cline{2-5}
& 1 & 0 & \cellcolor{red}1 & 1 \\ \cline{2-5}
\multicolumn{2}{c}{} & \multicolumn{2}{c}{d}
\end{tabular}




$a\overline{b}c\overline{d}$ абсолютно аналогичен.
Покрытие из четырех клеток включат в себя все остальные.

Этой группе соответствует терм $a\overline{b}$.

\begin{tabular}{c *{4}{|c}| c}
\multicolumn{1}{c}{} & \multicolumn{2}{c}{b} \\ \cline{2-5}
\multirow{2}{*}{a} & 1 & 1 & \cellcolor{pink}1 & \cellcolor{pink}1 \\ \cline{2-5}
& 0 & 1 & \cellcolor{pink}1 & \cellcolor{red}1 & \multirow{2}{*}{c} \\ \cline{2-5}
& 1 & 1 & 0 & 0 \\ \cline{2-5}
& 1 & 0 & 1 & 1 \\ \cline{2-5}
\multicolumn{2}{c}{} & \multicolumn{2}{c}{d}
\end{tabular}
\quad
\begin{tabular}{c *{4}{|c}| c}
\multicolumn{1}{c}{} & \multicolumn{2}{c}{b} \\ \cline{2-5}
\multirow{2}{*}{a} & 1 & 1 & 1 & 1 \\ \cline{2-5}
& 0 & 1 & \cellcolor{pink}1 & \cellcolor{red}1 & \multirow{2}{*}{c} \\ \cline{2-5}
& 1 & 1 & 0 & 0 \\ \cline{2-5}
& 1 & 0 & 1 & 1 \\ \cline{2-5}
\multicolumn{2}{c}{} & \multicolumn{2}{c}{d}
\end{tabular}
\quad
\begin{tabular}{c *{4}{|c}| c}
\multicolumn{1}{c}{} & \multicolumn{2}{c}{b} \\ \cline{2-5}
\multirow{2}{*}{a} & 1 & 1 & 1 & \cellcolor{pink}1 \\ \cline{2-5}
& 0 & 1 & 1 & \cellcolor{red}1 & \multirow{2}{*}{c} \\ \cline{2-5}
& 1 & 1 & 0 & 0 \\ \cline{2-5}
& 1 & 0 & 1 & 1 \\ \cline{2-5}
\multicolumn{2}{c}{} & \multicolumn{2}{c}{d}
\end{tabular}
\quad
\begin{tabular}{c *{4}{|c}| c}
\multicolumn{1}{c}{} & \multicolumn{2}{c}{b} \\ \cline{2-5}
\multirow{2}{*}{a} & 1 & 1 & 1 & 1 \\ \cline{2-5}
& 0 & 1 & 1 & \cellcolor{red}1 & \multirow{2}{*}{c} \\ \cline{2-5}
& 1 & 1 & 0 & 0 \\ \cline{2-5}
& 1 & 0 & 1 & 1 \\ \cline{2-5}
\multicolumn{2}{c}{} & \multicolumn{2}{c}{d}
\end{tabular}




Заметим, что описанные две группы и есть ядро.
Это следует из того, что для любой из оставшихся единиц покрытие максимального размера не содержит в себе всех меньших.

Для иллюстрации посмотрим на клетку $abcd$ и два её покрытия, изображенных ниже.
Большее покрытие, хоть и является единственной группой размера четыре, не входит в ядро, так как оно не содержит в себе изображенную группу размера два.

\begin{tabular}{c *{4}{|c}| c}
\multicolumn{1}{c}{} & \multicolumn{2}{c}{b} \\ \cline{2-5}
\multirow{2}{*}{a} & 1 & \cellcolor{pink}1 & \cellcolor{pink}1 & 1 \\ \cline{2-5}
& 0 & \cellcolor{red}1 & \cellcolor{pink}1 & 1 & \multirow{2}{*}{c} \\ \cline{2-5}
& 1 & 1 & 0 & 0 \\ \cline{2-5}
& 1 & 0 & 1 & 1 \\ \cline{2-5}
\multicolumn{2}{c}{} & \multicolumn{2}{c}{d}
\end{tabular}
\quad
\begin{tabular}{c *{4}{|c}| c}
\multicolumn{1}{c}{} & \multicolumn{2}{c}{b} \\ \cline{2-5}
\multirow{2}{*}{a} & 1 & 1 & 1 & 1 \\ \cline{2-5}
& 0 & \cellcolor{red}1 & 1 & 1 & \multirow{2}{*}{c} \\ \cline{2-5}
& 1 & \cellcolor{pink}1 & 0 & 0 \\ \cline{2-5}
& 1 & 0 & 1 & 1 \\ \cline{2-5}
\multicolumn{2}{c}{} & \multicolumn{2}{c}{d}
\end{tabular}




Посмотрим теперь на шесть оставшихся единиц, которые не покрыты ядром.
Можно заметить, что покрытие изображенное ниже оптимально.
Ему соответствует сумма $ad + \overline{c}\overline{d} + \overline{a}bc$.
В ней семь вхождений переменных.
Любое другое покрытие будет иметь восемь или более.

\begin{tabular}{c *{4}{|c}| c}
\multicolumn{1}{c}{} & \multicolumn{2}{c}{b} \\ \cline{2-5}
\multirow{2}{*}{a} & \cellcolor{cyan}1 & \cellcolor{pink}1 & \cellcolor{pink}1 & \cellcolor{cyan}1 \\ \cline{2-5}
& 0 & \cellcolor{pink}1 & \cellcolor{pink}1 & 1 & \multirow{2}{*}{c} \\ \cline{2-5}
& \cellcolor{lime}1 & \cellcolor{lime}1 & 0 & 0 \\ \cline{2-5}
& \cellcolor{cyan}1 & 0 & 1 & \cellcolor{cyan}1 \\ \cline{2-5}
\multicolumn{2}{c}{} & \multicolumn{2}{c}{d}
\end{tabular}

Поэтому минимальной ДНФ будет
\[a\overline{b} + ad + \overline{b}\overline{c} + \overline{c}\overline{d} + \overline{a}bc\]




\subsection*{2.1.4}

\textit{Построить логическую схему.}
\\

Схему рисовал в программе Logisim

\includegraphics[width=10cm]{2.1.4}

\newpage




\section*{2.2}

Задача во многом аналогична 2.1, поэтому комментарии только об отличающихся частях решения.

$f_1 = \{0, 1, 2, 3, 4, 5, 8, 9, 12, 13, 14\}$

Запишем таблицу истинности для $f$:

\begin{center}
	$\begin{array}{| c | c |}
	\hline
	abcd & f \\
	\hline
	0000 & 1 \\
	0001 & 1 \\
	0010 & 1 \\
	0011 & 1 \\
	0100 & 1 \\
	0101 & 1 \\
	0110 & 0 \\
	0111 & 0 \\
	1000 & 1 \\
	1001 & 1 \\
	1010 & 0 \\
	1011 & 0 \\
	1100 & 1 \\
	1101 & 1 \\
	1110 & 1 \\
	1111 & 0 \\
	\hline
	\end{array}$
\end{center}




\subsection*{2.2.1}

\textit{Записать СДНФ и СКНФ.}
\\

СДНФ:
\[\overline{a}\overline{b}\overline{c}\overline{d} + \overline{a}\overline{b}\overline{c}d + \overline{a}\overline{b}c\overline{d} + \overline{a}\overline{b}cd + \overline{a}b\overline{c}\overline{d} + \overline{a}b\overline{c}d + a\overline{b}\overline{c}\overline{d} + a\overline{b}\overline{c}d + ab\overline{c}\overline{d} + ab\overline{c}d + abc\overline{d}\]

СКНФ:
\[(a + \overline{b} + \overline{c} + d)(a + \overline{b} + \overline{c} + \overline{d})(\overline{a} + b + \overline{c} + d)(\overline{a} + b + \overline{c} + \overline{d})(\overline{a} + \overline{b} + \overline{c} + \overline{d})\]




\subsection*{2.2.2}

\textit{Минимизировать функцию с помощью метода Квайна-МакКласки для СКНФ.}
\\

Алгоритм Квайна-МакКласки - это вариация метода Квайна.
Основных изменений всего два.
Во-первых термы записываются последовательностью 0 и 1, а не переменных и отрицаний.
Во-вторых на этапе склейки термы группируются по количеству единиц.
Это убирает некоторые бесполезные сравнения.
\\

Выпишем все наборы, на которых $f = 0$ и пронумеруем их.

При этом используем цифровое представление вместо буквенного.
Например вместо $a + b + c + \overline{d}$ пишем $0001$.

Также сгруппируем наборы по количеству единиц.
\\

Проведем операцию \emph{склейки}.
Для этого ищем строки таблицы различающиеся только в одной переменной и склеиваем их.
Записываем результат в соседнюю таблицу, а сами строки помечаем, например серым фоном.

Аналогично методу Квайна склеиваем строки различающиеся только в одной переменной.
Легко заметить, что склеить можно только строки из соседних групп.
Т.е. те, у которых количество единиц различается ровно на один.

Вместо переменной по которой склеили ставим x, т.к. она для нас теперь не важна.

\begin{table}[h!]
\centering
$\begin{array}[t]{| c | c |} \hline
\rowcolor{lightgray}1 & 0110 \\
\rowcolor{lightgray}2 & 1010 \\ \hline
\rowcolor{lightgray}3 & 0111 \\
\rowcolor{lightgray}4 & 1011 \\ \hline
\rowcolor{lightgray}5 & 1111 \\ \hline
\end{array}$
\quad
\begin{tabular}[t]{| c | c |} \hline
3 - 5 & x111 \\ \hline
4 - 5 & 1x11 \\ \hline
1 - 3 & 011x \\
2 - 4 & 101x \\ \hline
\end{tabular}
\end{table}

Заполняем импликантная матрицу, расставляем галочки, обводим их, находим ядро и дополнение, записываем минимальнуу КНФ.
Все так же, как методе Квайна.
Не меняется ничего.

\begin{table}[h!]
\centering
	$\begin{array}{| c | *{5}{c} |}
	\hline
	& \cellcolor{pink}0110 & \cellcolor{pink}1010 & \cellcolor{pink}0111 & \cellcolor{pink}1011 & 1111 \\ \hline
	\text{x}111 & & & \checkmark & & \checkmark \\
	1\text{x}11 & & & & \checkmark & \checkmark \\
	\rowcolor{lime}011\text{x} & \textcircled{\checkmark} & & \checkmark & & \\
	\rowcolor{lime}101\text{x} & & \textcircled{\checkmark} & & \checkmark & \\ \hline
	\end{array}$
\end{table}

Ядро покрыло все, кроме последнего столбца.
В нем есть две галолчки, поэтому можно взять либо первую, либо вторую строки.

Таким образом в результате минимизации получается две тупиковых КНФ:

\[(a + \overline{b} + \overline{c})(\overline{a} + b + \overline{c})(\overline{a} + \overline{c} + \overline{d})\]
\[(a + \overline{b} + \overline{c})(\overline{a} + b + \overline{c})(\overline{b} + \overline{c} + \overline{d})\]




\subsection*{2.2.3}

\textit{Минимизировать функцию с помощью карт Карно для СКНФ.}
\\

Карты Карно подобны диаграммам Вейча.
Главное отличие - расположение переменных, и соответственно порядок строк/столбцов.

Т.к. мы минимизирум КНФ, то группировать будем нули.

\begin{table}[h!]
\centering
\begin{tabular}{c *{4}{|c} |}
\diagbox{$ab$}{$cd$} & 00 & 01 & 11 & 10 \\ \hline
00 & 1 & 1 & 1 & 1 \\ \hline
01 & 1 & 1 & 0 & 0 \\ \hline
11 & 1 & 1 & 0 & 1 \\ \hline
10 & 1 & 1 & 0 & 0 \\ \hline
\end{tabular}
\end{table}

Легко заметить, что две группы отмеченные ниже входят в ядро.

\begin{center}
\begin{tabular}{c *{4}{|c} |}
\diagbox{$ab$}{$cd$} & 00 & 01 & 11 & 10 \\ \hline
00 & 1 & 1 & 1 & 1 \\ \hline
01 & 1 & 1 & \cellcolor{cyan}0 & \cellcolor{cyan}0 \\ \hline
11 & 1 & 1 & 0 & 1 \\ \hline
10 & 1 & 1 & \cellcolor{lime}0 & \cellcolor{lime}0 \\ \hline
\end{tabular}
\end{center}

Соответствующие им множители в КНФ - это $(a + \overline{b} + \overline{c})$ и $(\overline{a} + b + \overline{c})$.
\\

Оставшийся ноль можно покрыть двумя равнозначными способами:

\begin{center}
\begin{tabular}{c *{4}{|c} |}
\diagbox{$ab$}{$cd$} & 00 & 01 & 11 & 10 \\ \hline
00 & 1 & 1 & 1 & 1 \\ \hline
01 & 1 & 1 & 0 & 0 \\ \hline
11 & 1 & 1 & \cellcolor{red}0 & 1 \\ \hline
10 & 1 & 1 & \cellcolor{pink}0 & 0 \\ \hline
\end{tabular}
\quad
\begin{tabular}{c *{4}{|c} |}
\diagbox{$ab$}{$cd$} & 00 & 01 & 11 & 10 \\ \hline
00 & 1 & 1 & 1 & 1 \\ \hline
01 & 1 & 1 & \cellcolor{pink}0 & 0 \\ \hline
11 & 1 & 1 & \cellcolor{red}0 & 1 \\ \hline
10 & 1 & 1 & 0 & 0 \\ \hline
\end{tabular}
\end{center}

Соответствующие множители: $(\overline{a} + \overline{c} + \overline{d})$ и $(\overline{b} + \overline{c} + \overline{d})$.
\\

Поэтому в результате минимизации получатся две тупиковых КНФ:

\[(a + \overline{b} + \overline{c})(\overline{a} + b + \overline{c})(\overline{a} + \overline{c} + \overline{d})\]
\[(a + \overline{b} + \overline{c})(\overline{a} + b + \overline{c})(\overline{b} + \overline{c} + \overline{d})\]




\subsection*{2.2.4}

\textit{Построить логическую схему.}
\\

Схему рисовал в программе Logisim

\includegraphics[width=10cm]{2.2.4_1}

\includegraphics[width=10cm]{2.2.4_2}

\newpage

\end{document}
