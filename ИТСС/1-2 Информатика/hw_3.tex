\documentclass[a4paper,12pt]{article}
\usepackage{ucs}
\usepackage{cmap}
\usepackage[utf8x]{inputenc}
\usepackage{amsfonts}
\usepackage[english,russian]{babel}
\usepackage[T1,T2A]{fontenc}
\frenchspacing
\usepackage{amsmath,amssymb,amsthm}
\usepackage[a4paper, margin=1in]{geometry}
\usepackage[table]{xcolor}
\usepackage{multirow}
\usepackage{diagbox}
\usepackage{graphicx}
\graphicspath{ {./images/} }
\usepackage{pgfplots}
\usepgfplotslibrary{fillbetween}
\usepackage{tikz-cd}

\begin{document}
\renewcommand{\arraystretch}{1.5}



\large
\begin{titlepage}
\begin{center}
ФЕДЕРАЛЬНОЕ ГОСУДАРСТВЕННОЕ АВТОНОМНОЕ ОБРАЗОВАТЕЛЬНОЕ УЧРЕЖДЕНИЕ ВЫСШЕГО ОБРАЗОВАНИЯ «НАЦИОНАЛЬНЫЙ ИССЛЕДОВАТЕЛЬСКИЙ УНИВЕРСИТЕТ «ВЫСШАЯ ШКОЛА ЭКОНОМИКИ»

\vspace{1cm}

Московский институт электроники и математики им. А.Н. Тихонова

\vspace{2cm}

Ефремов Виктор Васильевич, группа БИТ 203

\vspace{4cm}

Отчет

по домашней работе 2

\vspace{1cm}

по дисциплине "Информатика"

Тема: "Математические основы вычислительной техники"

\vspace{1cm}

Номер варианта: 6

Дата сдачи отчета: 11.12.2020

\vfill

Москва
2020
\end{center}
\end{titlepage}
\normalsize

\setcounter{page}{2}

\section*{Задание 1}

Канал связи задан двумя распределениями: $p(y / x)$ и $p(x)$.

\begin{table}[h]
$\begin{array}{c *{5}{|c} |}
& x_1 & x_2 & x_3 & x_4 & x_5 \\ \hline
p(x) & \frac{4}{19} & \frac{6}{19} & \frac{4}{19} & \frac{2}{19} & \frac{3}{19} \\ \hline
\end{array}$
\caption{$p(x)$}
\end{table}

\begin{table}[h]
$\begin{array}{c *{5}{|c} |}
& y_1 & y_2 & y_3 & y_4 & y_5 \\ \hline
x_1 & \frac{8}{9} & \frac{1}{9} & 0 & 0 & 0 \\ \hline
x_2 & \frac{3}{18} & \frac{5}{18} & \frac{4}{18} & 0 & \frac{6}{18} \\ \hline
x_3 & 0 & 0 & \frac{2}{8} & 0 & \frac{6}{8} \\ \hline
x_4 & \frac{9}{29} & \frac{9}{29} & \frac{1}{29} & \frac{6}{29} & \frac{4}{29} \\ \hline
x_5 & 0 & 0 & \frac{1}{16} & \frac{8}{16} & \frac{7}{16} \\ \hline
\end{array}$
\caption{$p(y / x)$}
\end{table}

Схематично нарисуем модель канала.
Стрелки показывают что можно получить на приемнике при отправке определенного сообщения.
Например, отправив $x_1$ можно получить либо $y_1$, либо $y_2$.

$$
\begin{tikzcd}
x_1 \arrow[rr] \arrow[drr] && y_1 \\
x_2 \arrow[urr] \arrow[rr] \arrow[drr] \arrow[dddrr] && y_2\\
x_3 \arrow[rr] \arrow[ddrr] && y_3 \\
x_4 \arrow[uuurr] \arrow[uurr] \arrow[urr] \arrow[rr] \arrow[drr] && y_4 \\
x_5 \arrow[uurr] \arrow[urr] \arrow[rr] && y_5 \\
\end{tikzcd}
$$

Для того чтобы найти взаимную информацию вспомним несколько формул.

\begin{gather*}
I(x, y) = H(y) - H(y / x) \\
H(y) = - \sum_{j} p(y_j) \cdot log_2 p(y_j) \\
H(y / x) = - \sum_{i} \sum_{j} p(x_i, y_j) \cdot log_2 p(y_j / x_i) \\
\end{gather*}

\begin{gather*}
p(y_j) = \sum_{i} p(x_i, y_j) \\
p(x, y) = p(x) \cdot p(y / x)
\end{gather*}

Для начала посчитаем $p(x, y)$.
Для этого нужно просто умножить каждый элемент матрицы $p(y / x)$ на соответствующую вероятность $p(x)$.
Например $p(x_1, y_1) = p(x_1) \cdot p(y_1 / x_1), p(x_1, y_2) = p(x_1) \cdot p(y_2 / x_1)$.

\begin{table}[h]
$\begin{array}{c *{5}{|c} |}
& y_1 & y_2 & y_3 & y_4 & y_5 \\ \hline
x_1 & \frac{32}{171} & \frac{4}{171} & 0 & 0 & 0 \\ \hline
x_2 & \frac{18}{342} & \frac{30}{342} & \frac{24}{342} & 0 & \frac{36}{342} \\ \hline
x_3 & 0 & 0 & \frac{8}{152} & 0 & \frac{24}{152} \\ \hline
x_4 & \frac{18}{551} & \frac{18}{551} & \frac{2}{551} & \frac{12}{551} & \frac{8}{551} \\ \hline
x_5 & 0 & 0 & \frac{3}{304} & \frac{24}{304} & \frac{21}{304} \\ \hline
\end{array}$
\caption{$p(x, y)$}
\end{table}

Найдем $p(y_j)$.
Для этого нужно просто просуммировать элемнты каждого столбца матрицы $p(x, y)$.

\begin{gather*}
p(y_1) = \frac{32}{171} + \frac{18}{342} + 0 + \frac{18}{551} + 0 = \frac{1351}{4959} \\
p(y_2) = \frac{4}{171} + \frac{30}{342} + 0 + \frac{18}{551} + 0 = \frac{713}{4959} \\
p(y_3) = 0 + \frac{24}{342} + \frac{8}{152} + \frac{2}{551} + \frac{3}{304} = \frac{3605}{26448} \\
p(y_4) = 0 + 0 + 0 + \frac{12}{551} + \frac{24}{304} = \frac{111}{1102} \\
p(y_5) = 0 + \frac{36}{342} + \frac{24}{152} + \frac{8}{551} + \frac{21}{304} = \frac{3057}{8816}
\end{gather*}

Сделаем проверку.
Сумма всех $p(y_i)$ должна быть равна единице, так как это полная группа событий.

$$\frac{1351}{4959} + \frac{713}{4959} + \frac{3605}{26448} + \frac{111}{1102} + \frac{3057}{8816} = 1$$
%1351/4959+713/4959+3605/26448+111/1102+3057/8816

\begin{table}[h]
$\begin{array}{c *{5}{|c} |}
& y_1 & y_2 & y_3 & y_4 & y_5 \\ \hline
p(y) & \frac{1351}{4959} & \frac{713}{4959} & \frac{3605}{26448} & \frac{111}{1102} & \frac{3057}{8816} \\ \hline
\end{array}$
\caption{$p(y)$}
\end{table}

\newpage

Посчитаем энтропию $H(y)$.

\begin{gather*}
H(y) = - \biggl( \frac{1351}{4959} \cdot log_{2} \frac{1351}{4959} + \frac{713}{4959} \cdot log_{2} \frac{713}{4959} + \frac{3605}{26448} \cdot log_{2} \frac{3605}{26448} + \\
\frac{111}{1102} \cdot log_{2} \frac{111}{1102} + \frac{3057}{8816} \cdot log_{2} \frac{3057}{8816} \biggr) \approx 2.1686847
\end{gather*}
%(1351/4959)*log2(1351/4959) + (713/4959)*log2(713/4959) + (3605/26448)*log2(3605/26448) + (111/1102)*log2(111/1102) + (3057/8816)*log2(3057/8816)

Посчитаем энтропию $H(y / x)$.

\begin{align*}
& H(y / x) = - \biggl( \frac{32}{171} \cdot log_{2} \frac{8}{9} + \frac{4}{171} \cdot log_{2} \frac{1}{9} + 0 + 0 + 0 + \\
& \frac{18}{342} \cdot log_{2} \frac{3}{18} + \frac{30}{342} \cdot log_{2} \frac{5}{18} + \frac{24}{342} \cdot log_{2} \frac{4}{18} + 0 + \frac{36}{342} \cdot log_{2} \frac{6}{18} + \\
& 0 + 0 + \frac{8}{152} \cdot log_{2} \frac{2}{8} + 0 + \frac{24}{152} \cdot log_{2} \frac{6}{8} + \\
& \frac{18}{551} \cdot log_{2} \frac{9}{29} + \frac{18}{551} \cdot log_{2} \frac{9}{29} + \frac{2}{551} \cdot log_{2} \frac{1}{29} + \frac{12}{551} \cdot log_{2} \frac{6}{29} + \frac{8}{551} \cdot log_{2} \frac{4}{29} + \\
& 0 + 0 + \frac{3}{304} \cdot log_{2} \frac{1}{16} + \frac{24}{304} \cdot log_{2} \frac{8}{16} + \frac{21}{304} \cdot log_{2} \frac{7}{16} \biggr) \approx 1.3137435
\end{align*}
%-0.10594912311 = (32/171)*log2(8/9) + (4/171)*log2(1/9)
%-0.6172692464 = (18/342)*log2(3/18) + (30/342)*log2(5/18) + (24/342)*log2(4/18) + (36/342)*log2(6/18)
%-0.17079539462 = (8/152)*log2(2/8) + (24/152)*log2(6/8)
%-0.21892204656 = (18/551)*log2(9/29) + (18/551)*log2(9/29) + (2/551)*log2(1/29) + (12/551)*log2(6/29) + (8/551)*log2(4/29)
%-0.2008077192 = (3/304)*log2(1/16) + (24/304)*log2(8/16) + (21/304)*log2(7/16)

Взяв разность найденных энтропий получим взаимную информацию.

$$I(x, y) \approx 2.1686847 - 1.3137435 \approx 0.8549412 \text{ бит}$$

\newpage

Посмотрим что будет при другом распределении вероятностей над входным алфавитом. Чтобы отличать новые распределения от старых будем обозначать их $\tilde{x}$ и $\tilde{y}$.
Пусть все сообщения источника равновероятны, т.е. все $p(\tilde{x}_i)$ равны.

\begin{table}[h]
$\begin{array}{c *{5}{|c} |}
& \tilde{x}_1 & \tilde{x}_2 & \tilde{x}_3 & \tilde{x}_4 & \tilde{x}_5 \\ \hline
p(\tilde{x}) & \frac{1}{5} & \frac{1}{5} & \frac{1}{5} & \frac{1}{5} & \frac{1}{5} \\ \hline
\end{array}$
\caption{$p(\tilde{x})$}
\end{table}

Аналогично предыдущему посчитаем совместную информацию $I(\tilde{x}, \tilde{y})$.

Посчитаем $p(\tilde{x}, \tilde{y})$.

\begin{table}[h]
$\begin{array}{c *{5}{|c} |}
& \tilde{y}_1 & \tilde{y}_2 & \tilde{y}_3 & \tilde{y}_4 & \tilde{y}_5 \\ \hline
\tilde{x}_1 & \frac{8}{45} & \frac{1}{45} & 0 & 0 & 0 \\ \hline
\tilde{x}_2 & \frac{3}{90} & \frac{5}{90} & \frac{4}{90} & 0 & \frac{6}{90} \\ \hline
\tilde{x}_3 & 0 & 0 & \frac{2}{40} & 0 & \frac{6}{40} \\ \hline
\tilde{x}_4 & \frac{9}{145} & \frac{9}{145} & \frac{1}{145} & \frac{6}{145} & \frac{4}{145} \\ \hline
\tilde{x}_5 & 0 & 0 & \frac{1}{80} & \frac{8}{80} & \frac{7}{80} \\ \hline
\end{array}$
\caption{$p(\tilde{x}, \tilde{y})$}
\end{table}

Посчитаем $p(\tilde{y})$.

\begin{table}[h]
$\begin{array}{c *{5}{|c} |}
& \tilde{y}_1 & \tilde{y}_2 & \tilde{y}_3 & \tilde{y}_4 & \tilde{y}_5 \\ \hline
p(\tilde{y}) & \frac{713}{2610} & \frac{73}{522} & \frac{2377}{20880} & \frac{41}{290} & \frac{2309}{6960}
\end{array}$
\caption{$p(\tilde{y})$}
\end{table}

Посчитаем $H(\tilde{y})$.

\begin{gather*}
H(\tilde{y}) = - \biggl( \frac{713}{2610} \cdot \log_2 \frac{713}{2610} + \frac{73}{522} \cdot \log_2 \frac{73}{522} + \frac{2377}{20880} \cdot \log_2 \frac{2377}{20880} + \\
\frac{41}{290} \cdot \log_2 \frac{41}{290} + \frac{2309}{6960} \cdot \log_2 \frac{2309}{6960} \biggr) \approx 2.1923048
\end{gather*}

%(713/2610)*log2(713/2610) + (73/522)*log2(73/522) + (2377/20880)*log2(2377/20880) + (41/290)*log2(41/290) + (2309/6960)*log2(2309/6960)

\newpage

Посчитаем $H(\tilde{y} / \tilde{x})$.

\begin{align*}
& H(\tilde{y} / \tilde{x}) = - \biggl( \frac{8}{45} \cdot log_2 \frac{8}{9} + \frac{1}{45} \cdot log_2 \frac{1}{9} + 0 + 0 + 0 + \\
& \frac{3}{90} \cdot log_2 \frac{3}{18} + \frac{5}{90} \cdot log_2 \frac{5}{18} + \frac{4}{90} \cdot log_2 \frac{4}{18} + 0 + \frac{6}{90} \cdot log_2 \frac{6}{18} + \\
& 0 + 0 + \frac{2}{40} \cdot log_2 \frac{2}{8} + 0 + \frac{6}{40} \cdot log_2 \frac{6}{8} + \\
& \frac{9}{145} \cdot log_2 \frac{9}{29} + \frac{9}{145} \cdot log_2 \frac{9}{29} + \frac{1}{145} \cdot log_2 \frac{1}{29} + \frac{6}{145} \cdot log_2 \frac{6}{29} + \frac{4}{145} \cdot log_2 \frac{4}{29} + \\
& 0 + 0 + \frac{1}{80} \cdot log_2 \frac{1}{16} + \frac{8}{80} \cdot log_2 \frac{8}{16} + \frac{7}{80} \cdot log_2 \frac{7}{16} \biggr) \approx 1.324153
\end{align*}

%(8/45)*log2(8/9) + (1/45)*log2(1/9)
%(3/90)*log2(3/18) + (5/90)*log2(5/18) + (4/90)*log2(4/18) + (6/90)*log2(6/18)
%(2/40)*log2(2/8) + (6/40)*log2(6/8)
%(9/145)*log2(9/29) + (9/145)*log2(9/29) + (1/145)*log2(1/29) + (6/145)*log2(6/29) + (4/145)*log2(4/29)
%(1/80)*log2(1/16) + (8/80)*log2(8/16) + (7/80)*log2(7/16)

Посчитаем $I(\tilde{x}, \tilde{y})$.

$$I(\tilde{x}, \tilde{y}) \approx 2.1923048 - 1.324153 \approx 0.8681518 \text{ бит}$$

Видно, что $I(\tilde{x}, \tilde{y}) > I(x, y)$.
Поэтому пропускная способность $C$ канала из задачи не меньше $I(\tilde{x}, \tilde{y}) \approx 0.8681518$. Но может быть и больше, т.к. $C = max_{p(x)} I(x, y)$ и возможно найдется распределение источника дающее ещё большую совместную информацию.

\newpage

\section*{Задание 2}

Используем следующую таблицу встречаемости букв (взято в руссокой википедии).

\renewcommand{\arraystretch}{1}
\begin{table}[h]
$\begin{tabular}{| c | c |}
\hline Буква & Частота, \% \\ \hline
а & 8.01 \\ \hline
б & 1.59 \\ \hline
в & 4.54 \\ \hline
г & 1.70 \\ \hline
д & 2.98 \\ \hline
е & 8.45 \\ \hline
ё & 0.04 \\ \hline
ж & 0.94 \\ \hline
з & 1.65 \\ \hline
и & 7.35 \\ \hline
й & 1.21 \\ \hline
к & 3.49 \\ \hline
л & 4.40 \\ \hline
м & 3.21 \\ \hline
н & 6.70 \\ \hline
о & 10.97 \\ \hline
п & 2.81 \\ \hline
р & 4.73 \\ \hline
с & 5.47 \\ \hline
т & 6.26 \\ \hline
у & 2.62 \\ \hline
ф & 0.26 \\ \hline
х & 0.97 \\ \hline
ц & 0.48 \\ \hline
ч & 1.44 \\ \hline
ш & 0.73 \\ \hline
щ & 0.36 \\ \hline
ъ & 0.04 \\ \hline
ы & 1.90 \\ \hline
ь & 1.74 \\ \hline
э & 0.32 \\ \hline
ю & 0.64 \\ \hline
я & 2.01 \\ \hline
\end{tabular}$
\caption{Встречаемость букв руссокго алфавита}
\end{table}

Информация, которая содержится в последовательности, суть сумма информаций каждого символа, которые равны $- log_2 p$.
ФИО - ефремов виктор васильевич.
Для удобства счета разобьем буквы по их количеству в сообщении.
в - 4; е - 3; и, о, р - 2; а, к, л, м, с, т, ф, ч, ь.

\newpage

Поэтому в моем ФИО содержится

\begin{gather*}
I = - \bigl( 4 \cdot log_2 0.0454 + 3 \cdot log_2 0.0845 + 2 \cdot (log_2 h0.0735 + log_2 0.01097 + log_2 0.0473) + \\
1 \cdot (log_2 0.0801 + log_2 0.0349 + log_2 0.0440 + log_2 0.0321 + log_2 0.0547 + \\
log_2 0.0626 + log_2 0.0026 + log_2 0.0144 + log_2 0.0174) \bigr) \approx 104.586 \text{ бит}
\end{gather*}
%4*log2(0.0454) + 3*log2(0.0845) + 2*(log2(0.0735) + log2(0.01097) + log2(0.0473)) + 1*(log2(0.0801) + log2(0.0349) + log2(0.0440) + log2(0.0321) + log2(0.0547) + log2(0.0626) + log2(0.0026) + log2(0.0144) + log2(0.0174)) 

Если закодировать все то же ФИО равномерным двоичным кодом, то на каждый сивол уйдет по 6 бит, т.к. $2^5 < 33 \leq 2^6$ и всего информации будет $6 \cdot 23 = 138$ бит.

\end{document}