\documentclass[a4paper,12pt]{article}
\usepackage{ucs}
\usepackage[utf8x]{inputenc}
\usepackage{amsfonts}
\usepackage[english,russian]{babel}
\usepackage[T1,T2A]{fontenc}
\frenchspacing
\usepackage{amsmath,amssymb,amsthm}
\usepackage[a4paper, margin=1in]{geometry}
\usepackage[table]{xcolor}
\usepackage{multirow}
\usepackage{diagbox}
\usepackage{graphicx}
\graphicspath{ {./images/} }

\newtheorem{name}{Printed output}
\newtheorem{problem}{Задача}
\newenvironment{solution}{\renewcommand{\proofname}{\unskip\indent\nopunct}\begin{proof}}{\end{proof}}

\begin{document}

\title{ДЗ 2}
\author{Витя\,Ефремов}
\maketitle

\begin{problem}
Вероятность события А равна 0.1, вероятность события B равна 0.2.
Условная вероятность события А при условии, что произойдет событие В, равна 0.25.
Найти вероятность суммы событий А и В.
Найти условную вероятность, что произойдет событие В, если известно, что произошло событие А.
\end{problem}
\begin{solution}
По определению
$$P(A \mid B) = \frac{P(AB)}{P(B)}$$
Откуда
$$P(A B) = P(A \mid B) P(B) = 0.25 \cdot 0.2 = 0.05$$
Из теоремы Байеса
$$P(B \mid A) = \frac{P(A \mid B) P(B)}{P(A)} = \frac{0.25 \cdot 0.2}{0.1} = 0.5$$
\end{solution}

\begin{problem}
Из колоды, содержащей 36 карт, достали две карты.
Выясните, зависимы ли события: А – одна из этих карт дама, другая – король, и В – обе эти карты пиковой масти.
\end{problem}
\begin{solution}
Вытащить даму можно $4$ способами, короля -- тоже. Любые комбинации возможны, поэтому
$$P(A) = \frac{4 \cdot 4}{C_{36}^{2}} = \frac{4 \cdot 4 \cdot 34! \cdot 2!}{36!} = \frac{8}{315} \approx 0.0254$$
Вероятность $B$ очевидна
$$P(B) = \frac{C_{9}^{2}}{C_{36}^{2}} = \frac{9! \cdot 34! \cdot 2!}{7! \cdot 2! \cdot 36!} = \frac{8 \cdot 9}{35 \cdot 36} = \frac{2}{35} \approx 0.0571$$
Пересечние событий $A$ и $B$ -- это событие ``вытащить пиковых даму и короля`` и его вероятность
$$P(AB) = \frac{1}{C_{36}^2} = \frac{34! \cdot 2!}{36!} = \frac{1}{630} \approx 0.00159$$
Легко видеть, что
$$P(AB) \neq P(A) P(B)$$
Откуда вывод -- события зависимы.
\end{solution}

\begin{problem}
Студент решает задачу по математике.
С вероятностью 0,15 он неправильно перепишет условие задачи.
Если он все же переписал его правильно, то с вероятностью 0,3 он выберет неправильный способ решения.
Если он выбрал правильный способ решения, то с вероятностью 0,2 он допустит ошибку в вычислениях.
Задача решена неверно.
Какова вероятность того, что студент выбрал неправильный способ решения?
\end{problem}
\begin{solution}
Обозначим три события $A$ -- ``неправильно переписал условие``, $B$ -- ``выбрал неправильный способ решения``, $C$ -- ``допустил ошибку в вычисленияях``.
Тогда вероятность решить задачу неверно это
$$P(A) + P(\overline{A})\left(P(B) + P(\overline{B}) P(C) \right) = 0.15 + 0.85 \cdot (0.3 + 0.7 \cdot 0.2) = 0.524$$
Вероятность выбрать неправильный способ решения это
$$P(\overline{A}) P(B) = 0.85 \cdot 0.3 = 0.255$$
Откуда итоговая вероятность
$$p = \frac{0.255}{0.524} \approx 0.4866$$
\end{solution}

\end{document}
