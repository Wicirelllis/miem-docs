\documentclass[a4paper,12pt]{article}
\usepackage{ucs}
\usepackage[utf8x]{inputenc}
\usepackage{amsfonts}
\usepackage[english,russian]{babel}
\usepackage[T1,T2A]{fontenc}
\frenchspacing
\usepackage{amsmath,amssymb,amsthm}
\usepackage[a4paper, margin=1in]{geometry}
\usepackage[table]{xcolor}
\usepackage{multirow}
\usepackage{diagbox}
\usepackage{graphicx}
\graphicspath{ {./images/} }

\newtheorem{name}{Printed output}
\newtheorem{problem}{Задача}
\newenvironment{solution}{\renewcommand{\proofname}{\unskip\indent\nopunct}\begin{proof}}{\end{proof}}

\begin{document}

\title{ДЗ 1}
\author{Витя\,Ефремов}
\maketitle


\begin{problem}
В продаже имеются фрукты: апельсины, лимоны, яблоки, персики, бананы.
Сколькими способами можно купить набор из 10 фруктов?
\end{problem}
\begin{solution}
Всего нужно 10 фруктов 5 разных видов. Т.к. порядок не важен и могут быть повторения, то используем формулу сочетаний с повторениями
$$C_{10+5-1}^{10} = C_{14}^{10} = \frac{14!}{4! \cdot 10!} = 1001$$
\end{solution}


\begin{problem}
Из двух групп выбирают по 2 студента для участия в олимпиаде. В 1-ой группе учатся 7 юношей и 5 девушек. Во 2-ой группе учатся 4 юноши и 11 девушек. Найти вероятность, что среди выбранных студентов будет 1 юноша и 3 девушки.
\end{problem}
\begin{solution}
Всего способов выбрать две пары людей (предполагая разных людей разными)
$$N = C_{7+5}^2 \cdot C_{4+11}^2 = 6930$$

Благоприятные случаи поделим на две части (П+Д из первой группы и Д+Д из второй, и наоборот) и посчитаем отдельно. Тогда
$$M_1 = 7 \cdot 5 \cdot C_{11}^2 = 1925$$
$$M_2 = C_5^2 \cdot 4 \cdot 11 = 440$$

Итоговая веротяность
$$p = \frac{M_1 + M_2}{N} = \frac{2365}{6930} \approx 0.34$$
\end{solution}


\begin{problem}
В стопке на полу в случайном порядке лежат 10 книг, среди которых имеются четыре тома романа “Война и мир”. Прежде, чем поставить книгу на полку, Федор Ридов ее прочитывает. Найти вероятность того, что после установки 6 книг Федор прочтет весь роман Л.Н. Толстого, причем в правильном порядке, но не обязательно подряд.
\end{problem}
\begin{solution}
Всего способов поставить книги в разном порядке на полку - это число перестановок
$$N = P_{10} = A_{10}^{10} = 10!$$

Благоприятных исходов
$$M = C_6^4 \cdot 6! = 15 \cdot 6!$$

Действительно, для четерех томов есть шесть мест. Используем сочетания, а не размещения, т.к. подходящий порядок только один. При этом расстановка остальных книг не важна, поэтому умножаем на $6!$.

Итоговая вероятность
$$p = \frac{M}{N} = \frac{15 \cdot 6!}{10!} = \frac{1}{336} \approx 0.003$$
\end{solution}
\end{document}
