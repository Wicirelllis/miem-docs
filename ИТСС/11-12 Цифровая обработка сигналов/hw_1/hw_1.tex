\documentclass[a4paper,12pt]{article}
\usepackage{ucs}
\usepackage[utf8x]{inputenc}
\usepackage{amsfonts}
\usepackage[english,russian]{babel}
\usepackage[T1, T2A]{fontenc}
\frenchspacing
\usepackage{amsmath,amssymb,amsthm}
\usepackage[a4paper, margin=1in]{geometry}
\usepackage{fontspec}
\usepackage{noto}
\usepackage[table]{xcolor}
\usepackage{multirow}
\usepackage{diagbox}
\usepackage{graphicx}
\usepackage{listings}
\usepackage{minted}
\usepackage[obeyspaces]{xurl}
\usepackage{hyperref}
\graphicspath{ {./} }
\renewcommand{\baselinestretch}{1.1}
\renewcommand{\arraystretch}{1.1}


\begin{document}

\sloppy

\title{ДЗ 1}
\author{Витя\,Ефремов}
\maketitle

\tableofcontents

\section{Теория}
Если говорить о цифровых сигналах, то их отличие от аналоговых в дискретности.
Дискретность по времени обычно называют дискретизацией.
Дискретность по ампллитуде - квантованием.
А цифровым считают сигнал дискретный и по времени, и по амплитуде.

Дискретизация сигнала - это семплирование, представление непрерывного сигнала/функции набором точек/значений.
Количество семплов в секунду - частота дискретизации.
Исторически стандартными частотами дискретизации являются 44.1 и 48 кГц.
Человек слышит звуки до примерно 20 кГц.
Из теоремы Котельникова следует, что частота дискретизации должна быть вдвое больше.
Откуда и получается частота чуть больше 40 кГц.
В \href{https://en.wikipedia.org/wiki/44,100_Hz}{вики} есть обсуждение других, менее значительных причин.

Тональный режим используется для передачи слежбной информации по телефонной сети.
Например, вызываемый номер.
Тональный режим позволяет кодировать 16 символов - \texttt{1234567890*\#ABCD}.
Каждый символ кодируется коротким звуковым сигналом состоящим из суммы двух синусоид.

\begin{table*}[ht]
    \centering
    \begin{tabular}{|c|c|c|c|c|}
        \hline & 1209 & 1336 & 1477 & 1633 \\
        \hline 697 & 1 & 2 & 3 & A \\
        \hline 770 & 4 & 5 & 6 & B \\
        \hline 852 & 7 & 8 & 9 & C \\
        \hline 941 & * & 0 & \# & D \\
        \hline
    \end{tabular}
    \caption{Частоты синусоид}
\end{table*}

Если сменить частоту дискретизации с дефолтных 10 кГц на 2 кГц, то разница заметна на слух.
Это происходит из-за того что частота дискретизации становится меньше удвоенной частоты высокого синуса.
Если же менять частоту дискретизации, оставляя её достаточно высокой, то изменений нет.
Т.е. что 10 кГц, что 8, что 15 звучат одинаково.

Так же график сигнала соответствующего восьмерке.
Только по нему тяжело сделать какие-то выводы, но если сравнить с графиками других цифр, то видно что они действительно отличаются.
\includegraphics*{/img/8.png}

\section{Приложение. Код}
Код также доступен в репозитории на гитхабе.
\url{https://github.com/Wicirelllis/miem-docs/blob/master/ИТСС/11-12 Цифровая обработка сигналов/hw_1/hw_1.py}

\inputminted[fontsize=\scriptsize]{python}{hw_1.py}

\end{document}
